\documentclass[conference]{IEEEtran}

%\usepackage[colorlinks=true,urlcolor=black,linkcolor=black,citecolor=black]{hyperref}
\usepackage{booktabs}
\usepackage{graphicx}

\usepackage{dcolumn}
\newcommand{\cc}[1]{\multicolumn{1}{c}{#1}}
\newcolumntype{d}[1]{D{.}{.}{#1}}

% use Lucida fonts for both text and math.
%\usepackage[altbullet]{lucidabr}     % get larger bullet
% or just use it for code:
%\renewcommand{\ttdefault}{hlst}
%% this one looks also better for courier:
\renewcommand*\ttdefault{txtt}

\usepackage{listings}
\lstset{
  basicstyle=\sffamily,
  keywordstyle=\sffamily,
  language=Java,
  columns=flexible, % add any modifications here
}

\lstnewenvironment{Java}{}{}

\newcommand{\code}[1]{{\textsf{#1}}}

% now we have colors ;-)
\usepackage[usenames,dvipsnames]{color}
\newcommand{\tommy}[1]{{\color{OliveGreen} Tommy: #1}}
\newcommand{\martin}[1]{{\color{blue} Martin: #1}}

\newcommand{\todo}[1]{\emph{TODO: #1}}

%\renewcommand{\tommy}[1]{}
%\renewcommand{\martin}[1]{}
%\renewcommand{\todo}[1]{}

\usepackage{textcomp} % for \textmu{s}

\widowpenalty=10000
\clubpenalty=10000

% for overhanging lines
\setlength{\emergencystretch}{3em}

\begin{document}

\title{YAMP: Yet Another MIPS Processor}

\author{\IEEEauthorblockN{Martin Schoeberl}
\IEEEauthorblockA{Department of Informatics and Mathematical Modeling\\
Technical University of Denmark\\
Email: masca@imm.dtu.dk}}



\maketitle \thispagestyle{empty}


\begin{abstract}

This document is a collection of observations during the design of a MIPS
compatible RSIC CPU for an FPGA.
\end{abstract}


\section{Introduction}

Although the MIPS described in the standard computer architecture book by
Hennessy and Patterson is described as 5 stage pipeline, the original MIPS
2000 has been a 4 stage pipeline [citation needed].

\section{Some Numbers}

Cyclone EP2C70-6 (Altera DE2-70 board), Quartus 10.1

Simple ALU with add, and, or, lui

PLL set to 200 MHz

with forwarding: 93 MHz
drop lui, or: 98 MHz
drop forwarding: 113 MHz
drop forwarding, and, or, lui: 122 MHz

Path goes from dec/address\_reg (RF) to exmem register destination register value.

Why is there a path from the RF address register to the ALU output register?

Remove forwarding within RF: Quartus writes that through path is added, 127 MHz

33 Logic levels. This is the adder, right?

Doing or instead of add: 203 MHz

MMh, is 32-bit adding so expensive?

%\section{Related Work}
%
%\section{Conclusions}
%\label{sec:conclusion}
%
%\section{Acknowledgments}
%
%\todo{This work received funding from XXX. }

% \bibliographystyle{plain} % similar to IEEE without URLs
\bibliographystyle{plain}
\bibliography{msbib}

\end{document}

